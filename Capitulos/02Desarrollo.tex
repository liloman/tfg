%---------------------------------------------------------------------
%
%                          Cap�tulo 2
%
%---------------------------------------------------------------------

\chapter{Desarrollo}

\begin{FraseCelebre}
\begin{Frase}
    La salud f�sica es el primer requisito para la felicidad.
\end{Frase}
\begin{Fuente}
    Joseph Hubertus Pilates 
\end{Fuente}
\end{FraseCelebre}

\begin{resumen}
En este capitulo  ..
\end{resumen}


\section{Introducci�n}
\label{cap2:sec:introduccion}


\subsection{Prop�sito}

Como se dijo en  \nameref{cap1:sec:escenario} o por su n�mero \ref{cap1:sec:escenario}.

Para incluir una tabla (\nameref{cap2:tab:menganitos})...


\begin{table}[H]
    \centering %\footnotesize
    \begin{tabularx}{0.9\textwidth}{$X^X}
        \toprule \rowstyle{\bfseries}
        Valor & Diferencia \\
        \toprule
        <= 1  & No perceptible por los ojos humanos  \\
        1 - 2  & Perceptible bajo observaci�n cercana   \\
        2 - 10 & Perceptible en un vistazo \\
        11 - 49  &  Los colores son m�s similares que diferentes  \\
        100   &  Los colores son exactamente contrarios \\
        \bottomrule 
    \end{tabularx}
    \caption{Valores de menganitos}
    \label{cap2:tab:menganitos}
\end{table}

\subsection{Mas cosas}


Como se dijo en \nameref{cap1:sec:escenario}...

\section{Fin}

Para incluir una figura simple:

\figuraFija{Vectorial/Capitulos/2/sola}{width=0.8\textwidth}{capitulos:fig:2/sola}{ Algo asi como una imagen sola}

Y todo lo demas ya es cosa de ir poco a poco pero esto da para bastante...



\medskip
